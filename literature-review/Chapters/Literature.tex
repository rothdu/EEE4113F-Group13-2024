% ----------------------------------------------------
% Literature Review
% ----------------------------------------------------
\documentclass[class=report,11pt,crop=false]{standalone}
% Page geometry
\usepackage[a4paper,margin=20mm,top=25mm,bottom=25mm]{geometry}

% Font choice
\usepackage{lmodern}

% \usepackage{lipsum}

% Use IEEE bibliography style
\bibliographystyle{IEEEtran}

% Line spacing
\usepackage{setspace}
\setstretch{1.20}

% Ensure UTF8 encoding
\usepackage[utf8]{inputenc}

% Language standard (not too important)
\usepackage[english]{babel}

% Skip a line in between paragraphs
\usepackage{parskip}

% For the creation of dummy text
\usepackage{blindtext}

% Math
\usepackage{amsmath}

% Header & Footer stuff
\usepackage{fancyhdr}
\pagestyle{fancy}
\fancyhead{}
\fancyhead[R]{\nouppercase{\rightmark}}
\fancyfoot{}
\fancyfoot[C]{\thepage}
\renewcommand{\headrulewidth}{0.0pt}
\renewcommand{\footrulewidth}{0.0pt}
\setlength{\headheight}{13.6pt}

% Epigraphs
\usepackage{epigraph}
\setlength\epigraphrule{0pt}
\setlength{\epigraphwidth}{0.65\textwidth}

% Colour
\usepackage{color}
\usepackage[usenames,dvipsnames]{xcolor}

% Hyperlinks & References
\usepackage{hyperref}
\definecolor{linkColour}{RGB}{77,71,179}
\hypersetup{
    colorlinks=true,
    linkcolor=linkColour,
    filecolor=linkColour,
    urlcolor=linkColour,
    citecolor=linkColour,
}
\urlstyle{same}

% Automatically correct front-side quotes
\usepackage[autostyle=false, style=ukenglish]{csquotes}
\MakeOuterQuote{"}

% Graphics
\usepackage{graphicx}
\graphicspath{{Images/}{../Images/}}
\usepackage{makecell}
\usepackage{transparent}

% SI units
\usepackage{siunitx}

% Microtype goodness
\usepackage{microtype}

% Listings
\usepackage[T1]{fontenc}
\usepackage{listings}
\usepackage[scaled=0.8]{DejaVuSansMono}

% Custom colours for listings
\definecolor{backgroundColour}{RGB}{250,250,250}
\definecolor{commentColour}{RGB}{73, 175, 102}
\definecolor{identifierColour}{RGB}{196, 19, 66}
\definecolor{stringColour}{RGB}{252, 156, 30}
\definecolor{keywordColour}{RGB}{50, 38, 224}
\definecolor{lineNumbersColour}{RGB}{127,127,127}
\lstset{
  language=Matlab,
  captionpos=b,
  aboveskip=15pt,belowskip=10pt,
  backgroundcolor=\color{backgroundColour},
  basicstyle=\ttfamily,%\footnotesize,        % the size of the fonts that are used for the code
  breakatwhitespace=false,         % sets if automatic breaks should only happen at whitespace
  breaklines=true,                 % sets automatic line breaking
  postbreak=\mbox{\textcolor{red}{$\hookrightarrow$}\space},
  commentstyle=\color{commentColour},    % comment style
  identifierstyle=\color{identifierColour},
  stringstyle=\color{stringColour},
   keywordstyle=\color{keywordColour},       % keyword style
  %escapeinside={\%*}{*)},          % if you want to add LaTeX within your code
  extendedchars=true,              % lets you use non-ASCII characters; for 8-bits encodings only, does not work with UTF-8
  frame=single,	                   % adds a frame around the code
  keepspaces=true,                 % keeps spaces in text, useful for keeping indentation of code (possibly needs columns=flexible)
  morekeywords={*,...},            % if you want to add more keywords to the set
  numbers=left,                    % where to put the line-numbers; possible values are (none, left, right)
  numbersep=5pt,                   % how far the line-numbers are from the code
  numberstyle=\tiny\color{lineNumbersColour}, % the style that is used for the line-numbers
  rulecolor=\color{black},         % if not set, the frame-color may be changed on line-breaks within not-black text (e.g. comments (green here))
  showspaces=false,                % show spaces everywhere adding particular underscores; it overrides 'showstringspaces'
  showstringspaces=false,          % underline spaces within strings only
  showtabs=false,                  % show tabs within strings adding particular underscores
  stepnumber=1,                    % the step between two line-numbers. If it's 1, each line will be numbered
  tabsize=2,	                   % sets default tabsize to 2 spaces
  %title=\lstname                   % show the filename of files included with \lstinputlisting; also try caption instead of title
}

% Caption stuff
\usepackage[hypcap=true, justification=centering]{caption}
\usepackage{subcaption}

% Glossary package
% \usepackage[acronym]{glossaries}
\usepackage{glossaries-extra}
\setabbreviationstyle[acronym]{long-short}

% For Proofs & Theorems
\usepackage{amsthm}

% Maths symbols
\usepackage{amssymb}
\usepackage{mathrsfs}
\usepackage{mathtools}

% For algorithms
\usepackage[]{algorithm2e}

% Spacing stuff
\setlength{\abovecaptionskip}{5pt plus 3pt minus 2pt}
\setlength{\belowcaptionskip}{5pt plus 3pt minus 2pt}
\setlength{\textfloatsep}{10pt plus 3pt minus 2pt}
\setlength{\intextsep}{15pt plus 3pt minus 2pt}

% For aligning footnotes at bottom of page, instead of hugging text
\usepackage[bottom]{footmisc}

% Add LoF, Bib, etc. to ToC
\usepackage[nottoc]{tocbibind}

% SI
\usepackage{siunitx}

% For removing some whitespace in Chapter headings etc
\usepackage{etoolbox}
\makeatletter
\patchcmd{\@makechapterhead}{\vspace*{50\p@}}{\vspace*{-10pt}}{}{}%
\patchcmd{\@makeschapterhead}{\vspace*{50\p@}}{\vspace*{-10pt}}{}{}%
\makeatother

% Tables
\usepackage{multirow}
\usepackage{longtable}
\usepackage{tabularx}
\makenoidxglossaries

\newacronym{radar}{RADAR}{Radio Detection and Ranging}
\begin{document}
\ifstandalone
\tableofcontents
\fi
% ----------------------------------------------------
\chapter{Literature Review \label{ch:literature}}
% \epigraph{}%
    % {\emph{---Carl Sagan}}
% \vspace{0.5cm}
% ----------------------------------------------------

\section{An overview of existing full camera trap systems}


Swann et al. \cite{swann2011evaluating} argue that while choosing from a selection of camera traps designed for generalised use cases may be complex, it becomes essential to refine a module designed for nest ecology researchers. Meek et al. \cite{meek2012user}, emphasise the key elements such as cost effectiveness, speed, space and weight saving, usability, and versatility as being critical in developing an efficient wildlife surveillance system.  
% took out a few words about infrared - it's covered in the motion detection section

Recent studies highlight the ongoing efforts to improve camera trap systems in wildlife research by addressing issues of detection reliability, data cataloguing, and remote communication capabilities, while also revealing challenges in power consumption, weight, and cost. Nazir, Sajid, et al. \cite{nazir2017wiseeye} introduced the “WiseEye” camera trap system, which aimed to address frustrations of existing camera systems. The “WiseEye” system, consisting of a Raspberry Pi and passive infrared sensors, promised and showed an improvement in detection reliability, reduced data cataloguing time, and remote communication capabilities. With this solution came many drawbacks such as lower power efficiency, weight, and cost. This indicates potential areas that can be simplified and optimised to suit nest ecology research. Rico-Guevara et al. \cite{rico-guevara2017bring} contributed by confirming the importance of having a PIR sensor in their study that involved capturing pictures of hummingbirds. They also had issues when it came to power consumption and weight. This is a recurring issue among existing designs. Green, Sian, et al. \cite{green2020innovations} explored the integration of artificial intelligence with existing camera trap solutions, in a bid to offer advancements in data processing and analysis. This brought upon its own issues such as increased power consumption and greater computational requirements that would only be fixed if the whole system is redesigned. 


% maybe take this line out
Additionally, Rovero et al. \cite{rovero2013which} emphasises the importance of establishing a selection criterion when choosing amongst off-the-shelf products, especially in niche environments like nesting areas.

% possibly into night vision section
Aguiar-Silva, Francisca Helena, et al. \cite{aguiar-silva2017camera} agreed on the use of infrared sensors because of the prevalence of nocturnal predators that attack nesting areas of birds. This will make it easier to analyse activity regardless of night or day in confined areas.

In a study regarding nest predation, Ribeiro-Silva et al. \cite{ribeiro-silva2018testing} again stress the importance of infrared cameras but warn against positioning them too far away from the nest to reduce overexposure of the captured footage.

Wireless data transmission and automation emerge as crucial requirements in modern camera trap modules, according to Glover-Kapfer et al. \cite{glover2019camera}. However, they make note of the fact that availability of good camera traps, i.e., those that are relevant to the researchers’ requirements, is often dictated by the novelty task at hand, which may limit the available options for studies in nest ecology.  

\subsection{Challenges and limitations of consumer camera traps}

Addressing operational challenges and efficiency issues in nest monitoring systems is important when designing effective wildlife research modules, as indicated by Ribeiro-Silva  et al. \cite{ribeiro-silva2018testing}, identify various shortcomings in existing solutions and proposed strategies for improvement. In the pursuit of effective nest monitoring, \cite{ribeiro-silva2018testing} found that problems like insufficient storage, wind displacement, weather damage, and false detection of animals outside the intended field of view led to poor efficiency, loss of data, and missed predation lists. Suggested improvements from \cite{ribeiro-silva2018testing} include camera positioning, increased storage, regular inspection, and the use of more than one camera to progressively monitor activity around the nest. 

\cite{ahumada2020wildlife} suggest that existing camera trap modules may not meet the standards that are expected for scientific researcg and conservation. \cite{ahumada2020wildlife} stresses the necessity of laying groundwork to establish hardware standards tailored to each study, challenging the feasibility of adopting off-the-shelf modules for diverse researching tasks. 

%is the first sentence the from Glover or Ribeiro-silva?
A Bushnell TrophyCam, which captures both videos and photos, was found to provide intuitive data for distinguishing between different birds within the same species through analysing their behaviour from short video intervals. Challenges experienced involved low detection efficiencies with motion sensors and a need for managing the video lengths to optimise data storage \cite{ribeiro-silva2018testing}. 

% I think a short bit about the lynx study (with the 6 different modules) might be a good addition here

\section{Data transmission techniques}

\subsection{Comparison of wired and wireless data transmission}

Despite the improvements of wireless communication capabilities, wired communications have traditionally served as the industrty standard. Huynh’s study on general communication protocols \cite{huynh2010study} sheds light on several key advantages of wired data transmission. These include implementation of parallel data transfer techniques, with the speed can be adjusted by the number of parallel cables, provided the transfer rate per cable is maintained and the cables are shieled to minimise cross talk. \cite{huynh2010study} also made note of differential lines being utilised instead of single-ended lines, to improve data speeds, lower the power consumption and reduce electromagnetic interference, all pertinent to the scope of the design.

Gula et al. suggest that wired solutions may introduce problems such as cable management and animal interference. \cite{gula2010audio} details how different coaxial cable thicknesses are required to transmit different data types over a set length without any additional amplification, resulting in a cluster of cables running between subsystems. \cite{gula2010audio} also noted concerns that animals, such as rats, end up chewing the cables over time. 

Given the challenges associated with wired communications in wildlife research, wireless data transmission becomes a promising alternative. Adsumilli, C et al.  mention that ‘error resilience and error concealment were the most important aspects of current research for successful realization of transmission and reception of image/video signals over bandwidth limited fading wireless networks/channels.’ \cite{adsumilli2002adaptive}. % the quote identifies a problem area for wireless - does it fully back up the topic sentence?

Li et al. \cite{li2010design} suggest that implementing a wireless communication system will present a more cost-effective solution for wildlife research.  \cite{li2010design} further stipulares that deploying a wireless system will reduce the overhead that comes with maintenance when the connections get damaged.

\cite{adsumilli2002adaptive} H.263 and MPEG-4 encoding standards, which were implemented to achieve error control for both reception and transmission of video in error prone environments \cite{adsumilli2002adaptive}.

In exploring alternative wireless connectivity modules for low power IoT devices, Peng, Y. et al. \cite{peng2018plora} put forward a solution that is suitable for long-range transmission of low-rate data, which offers a simplification of camera modules in wildlife research. The study goes into the wireless device designed for batteryless IoT devices (passive devices), that are capable of transmitting data over long distance. While this is unsuitable for video or image transmission, the device will allow for the user to wirelessly obtain logs on the status of the module even when power has run out. \cite{peng2018plora} reported a maximum packet detection range of 50 m, making it a possible solution for smaller data transmission needs. % what is the solution?

Conversations with the researchers for the project at hand \cite{hofmeyer2024private} WiFi-based communication may be hampered by a lack WiFi connectivity at many of the red-winged starling nesting sites.

\section{Integration of camera traps with the Internet of Things (IOT)}

Single-board microcomputers can be a valuable tool for biological research. Jolles argues that "[T]he Raspberry Pi [is] a great reserach tool that can be used for almost anything. This can range from ... video recording of laboratory experiments, to long term field measurement stations" \cite{jolles2021broad-scale}. The wide versatility of a device like the Raspberry Pi makes it ideal for implementation of customisable video monitoring solutions. Jolles \cite{jolles2021broad-scale} also argues that the small footprint of the Rasbperry Pi aids in its use in research, and versatile options for the provision of power allow it to be used with minimal human intervention over long time periods. These benefits align strongly with the needs of the project at hand, further emphasizing the Raspberry Pi as a viable platform around which to base the project.

Data storage and retrieval has been identified as a key limitation in the application of camera traps to monitor Red-Winged starlings \cite{hofmeyer2024private}. Prinz et al. \cite{prinz2016a} implement a project for video surveillance of Acorn Woodpecker nests. They justify the Raspberry Pi as useful resource owing to its small size and variety of storage and connectivity options. The particular project uses a WiFi connection to upload footage to cloud storage on Google Drive. They further identify that the onboard SD card has sufficient storage for around 3 days worth of footage, but also identify the strength of the Raspberry Pi in its ability to leverage 3G/4G connectivity or external flash drives and hard drives as alternative means of data storage.  The variety of footage collection and/or storage options make it likely that a suitable option can be found in the case of Red-Winged starlings.

\cite{jolles2021broad-scale} acknowledges the existance of other microcomputer systems, argues that the use of these systems is hampered by lack of hardware and software maintanence, alongside generally poorer user support compared to the widespread Raspberry Pi. 

Microcontrollers, including options from ST Microelectronics, Espressif, and Atmel offer an alternative to full microcomputers. Camacho et al. \cite{camacho2017deployment} discuss a custom design based on the ATMEGA2650 microcontroller. They use a customised circuit design to allow for advanced power management, with the goal of limiting power draw. \cite{camacho2017deployment} further describe difficulties with using commonly available prototyping board, which are not necessarily as capable of achieving power draw requirements as custom circuit designs. Cardoso et al. \cite{cardoso2022internet} implement a remote visual surveillance based on an ESP32 microcontroller, demonstrating a successful microcontroller-based implementation under slightly different circumstances to \cite{camacho2017deployment}. \cite{jolles2021broad-scale} also identifies that microcontrollers may have even less stringent power requirements than the Raspberry Pi and argues that simple, repetitive tasks are often better suited to microcontrollers. Microcontroller based solutions may offer more customisability and lower costs compared to microcomputer solutions such as the Raspberry Pi.

 \cite{jolles2021broad-scale} also identifies that microcontrollers and microcomputers can be used in tandem in some instaces. This would allow the benefits of both systems to be realised, such as allowing for exceptionally low power draw the majority of the time, but allowing for the advanced connectivity options of the Raspberry Pi when required.

Both microcontroller and single-board microcomputers have varying price, typically depending on processing power and available peripherals. Local market prices may make higher-end devices too costly to be viable, and microcontrollers are generally significantly cheaper than microcomputers.

\section{Motion detection for camera traps}

Meek and Pittet \cite{meek2012user} identify that "Motion detection is of prime importance for any camera trap to be effective". Importantly, Rovero et al. \cite{rovero2013which} discuss that the camera field of view does not correspond directly to the detection zone. Motion detection is often a separate process from video or image recording, and must be thoroughly considered to make for a viable solution.

Meek and Pittet \cite{meek2012user} identify that most commercial camera traps use passive infrared (PIR) sensors to detection motion of wildlife, but are prone to false detections, resulting in images that don't feature any wildlife. The sensitivity of a PIR motion detector "depends on many factors, including the distance of the moving target, the temperature differential, the size of the animal, its speed and background light." Rovero et al. \cite{rovero2013which} add to this by identifying that PIR sensors can be triggered by pockets of hot air or vegetation. Meek et al. \cite{meek2012introduction} support a similar conclustion, indicating that "Where the temperature differential between the background and target is low, some sensors may be incapable of detecting target animals right in front of the camera". The prevalence of PIR based motion detection seems to indicate that it is a workable and useful solution, but evidence shows that there are also difficulties associated with the technology.

Rico-Guevara and Mickley \cite{rico-guevara2017bring} chose to use PIR motion sensors because they "were cheap and easy to deploy, and successfully detected hummingbirds." Their study found that detection of even the smallest species of hummingbirds was adequate for their application. They chose to combat the issue of false detections by intentionally positioning the sensors away from objects that could trigger unwanted detections, and mitigated the consequently limited scope of an individual sensors by triggering from multiple differently positioned sensors. A separate standalone trigger was also developed which included an adjustable sensistivity, allowing "fine tuning by optimizing the trade-off between increased sensitivity and false positives" \cite{rico-guevara2017bring}.

Welbourne et al \cite{welbourne2016how} identify that the functioning of PIR sensors is often misunderstood, leading to "flawed inferences or expecatations of camera performance." A thourough understanding of PIR sensors would help in the development of a robust and well-functioning solution.

Rovero et al. \cite{rovero2013which} discuss that older comercial camera traps from the 1980s were triggered by a break in an infrared beam. However, newer models come in "self-contained package including sensors and camera." % comment on the hobbs solution here

Prinz et al. \cite{prinz2016a} use an alternative technique which detects changes in the pixels from the camera output. This was achieved using the open-source program motion-MMAL. This alternative technique minimises the hardware requirements of the system.

% definitely still need to integrate above and below

\section{Sensing: ruviel}

% could maybe go right at the beginning?

The first step of action is to understand our subject of matter in much more detail so we can accurately gain a better understanding of how motion detection can be integrated into the environment of Starling birds. According to Ricklefs \cite{ricklefs1968patterns}, a starling bird can reach full body growth in approximately 20 days and an adult starling can reach a body mass of 79 grams. \cite{ricklefs1968patterns} further details that a newly hatched starling chick can weigh 5.27 grams and can be one of up to 18 chicks in brood in the starling nest. 

Verstraeten et al. \cite{verstraeten2010webcams} explain that the monitoring of birds has widespread potential in numerous applications in ecology, climatology, and avian related zoonosis/infections such as avian influenza. Hence, it is fundamentally important to make sure that monitoring these Starlings can be done in the most efficient and accurate way possible.

According to Amusa \cite{amusa2015pyro}, there are three technologies are used in practice for motion detection; they are Passive InfraRed (PIR), ultrasonic and audible sensors. In some instances, researchers developed motion detectors for birds, so that they can detect and repel the birds. However, in our case we take interest in just the detection of the birds. Siahaan \cite{siahaan2017design} designed an electronic device for detecting and repelling pests using pyro-electric (PIR) sensors.

\subsection{Existing sensors in bird monitoring setups}

There are various solutions to monitoring birds. These different types are listed and discussed in much greater detail below.

\subsection{Web cameras}

Web cameras are used all around the world, in all sorts of daily activities such as surveillance. According to Pritch \cite{pritch2007webcam}, it is reported that in the UK alone there are 4.2 million security cameras covering city streets. This indicates that web cameras are very useful when it comes to monitoring. However, there are downsides to the web cameras as well. Rateledge \cite{ratledge2005introduction} explains that webcams unedited raw data [6]. This is not good as it will have recordings of the entire day which is a waste of space and makes it tedious finding key events.

\subsection{Pyro-electric (PIR) sensors}

Pyroelectric Infrared (PIR) sensors are devices that detect infrared radiation emitted by objects in their field of view. Yun \cite{yun2014detecting} explains that they work based on the principle of pyroelectricity, where certain materials generate a voltage when exposed to changes in temperature. \cite{amusa2015pyro} details that PIR sensors have found extensive use across diverse fields, particularly in the creation of intelligent settings like healthcare facilities, smart energy systems, and security setups.


\section{Night or low-light visibility}

Rovero at al. \cite{rovero2013which} stipulate that camera trap modules sometimes infrared or white LED flash, to enable camera visibility at night. Flash technology is not generally needed when daylight can provide the necessary visibility for photography.

According to Meek et al. \cite{meek2012introduction}, infrared flash tend to be less disruptive to wildlife than incadecent white flash, alongside exhibiting superior power efficiency and faster overall camera trap trigger speeds. However, \cite{meek2012introduction} also explained that infrared flash are "are often in grey-scale ... or may have a
reddish-pink tinge.", whereas white flashes allow for colour images, which is sometimes useful for research. \cite{meek2012introduction} identifies that LED technologies for white flashes can mitigate concerns about power draw and trigger speed. At the time of publication of \cite{meek2012introduction}, only one commercial camera trap module used LED flash, but LED technology has since become much more accessibly, making it a much more viable solution than it previously was.

Conversations with the researchers working with Red Winged starlings \cite{hofmeyer2024private} indicated that infrared flash may be useful for night-time visibility of nests, or for enhanced visibility of some nesting sites that are very sheltered from daylight.


\section{Power supply}


Power plays in integral part, by supplying the entire design with the necessary power to run the device efficiently and adequately. The aim of this research is to determine the best possible approach in order to making the power submodule of the design as optimal as it can be.

\subsection{Power sources for sensors situated in an impractical setting}

There are various power sources that can be used, specifically when it come to remote sensors for lower power consumption applications. Dewan et al. \cite{dewan2014alternative} explain that one of the main issues with deploying sensors in remote or impractical environments is the life span of these sensors which is largely dependent on the power supply it is connected to. With this in mind we can look into the various power sources that could be suitable for our design that will follow.

\subsection{Battery powered supply}

Polleneli et al. \cite{polonelli2020flexible} describe the success of using batteries to power their motherboard for their drone application. They further iterated that their design included generating reliable and stable 3.3 V and 5 V DC voltage from two Panasonic 18650 batteries which managed fast charging. This can be tailored to our design as it has the capabilities of fast charging. Callebaut et al. \cite{callebaut2021art} outlines issues faced with batteries and their lifespans and how to lengthen their life. They further explained the different types of batteries and how the battery type can optimise the battery life. They stated, “Lithium Cobalt Oxide (LCO) and Lithium Polymer (LIPO) batteries are currently the most used rechargeable technologies” \cite{callebaut2021art}. These batteries can be useful and go hand in hand with solar power supply.

\subsection{Solar powered supply}

The popularity and extensive use of solar is explaiend through the statement, “Solar power is used for operating both ground and floating sensors; solar-powered sensors have been deployed in forests, on mountains, in deserts, on ocean surfaces, on islands, and in household and industrial buildings.” \cite{dewan2014alternative}. This supports the possibility of integrating solar power into the design in order to reduce the number of manual recharges of the battery required. Krejcar further supports the use of solar powered supply for remote sensors and explains that “It is possible to apply the solution from lower voltage (lower sun intensity)” \cite{krejcar2012optimized}, this indicates that it will be able to charge the battery and operate on days that are not extremely sunny.

\subsection{Power saving operations for sensors situated in an impractical setting}

Power consumption is an important aspect to think about when it comes to design and it is very important that power saving operations are put into place in order to make the whole device more efficient. Some of these methods are to be studied and analysed below.

Harikrishnan and Sivagami \cite{harikirshnan2017intelligent} go into detail talking about how PIR sensors can be used to switch a system on and off depending on weather a sensor is giving a reading or not [5]. This design can be extremely useful in our design as it will switch the device on only when there is motion, hence reducing redundant recordings and saving power. Harsha and Kumar \cite{harsha2020home} further support this notion and show the justification of the use of a power saving through tests.

% ----------------------------------------------------
\ifstandalone
\bibliography{../Bibliography/References.bib}
\printnoidxglossary[type=\acronymtype,nonumberlist]
\fi
\end{document}
% ----------------------------------------------------