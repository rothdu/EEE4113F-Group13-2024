\chapter{Mechanical Design}


\section{Introduction}
This section covers the mechanical case designs crafted for the the remote surveillance system deployed at the the Starlings’ nests located within the University of Cape Town for Sally who works at the Fitz Patrick Institute. The goal for this project was enhance nest surveillance through innovative technological solutions.  These enclosures serve as the protective shell for our electronic components, safeguarding them against the elements while ensuring optimal functionality. By providing robust protection against environmental factors such as moisture, dust, and temperature fluctuations, the mechanical cases play a crucial role in the reliability and longevity of the surveillance system.
The surveillance system addresses existing challenges and gaps in current methods of monitoring bird nests by providing real-time data collection, minimizing disturbance to nesting birds, and enabling remote observation capabilities. Traditional nest monitoring methods often rely on manual observation, which can be time-consuming, labour-intensive, and disruptive to nesting birds. Our system offers a non-invasive alternative that allows researchers to gather valuable data without causing undue stress to the bird population.
The mechanical cases discussed in this section will house the Transmitter and Receiver circuitry systems that aim to solve the surveillance of the nests. This section discusses the design decisions, material choices and processes undertaken to realise the ‘final’ prototype. 
A comprehensive breakdown of the design processes, material choices, and iteration phases involved in the development of the mechanical cases. Exploring the intricacies of each design iteration, discussing the rationale behind design decisions and the challenges encountered during the prototyping process. Additionally,  results fro, testing and evaluation efforts, offering insights into the performance and effectiveness of the 'final' prototype. Finally, we will draw conclusions based on our findings and discuss potential avenues for future research and development in the field of avian monitoring technology.

Requirements, Specifications, and acceptable prototype
•	Size: 6x6x8cm
•	Receiver case: handheld capability or mounting option
•	Transmitter and receiver cases: safely house the sensor and power module circuits
•	Circuit protection from environmental conditions and theft prevention
•	Ease of access for maintenance and user experience
•	Compatibility with monitoring equipment
Design Process
First Iteration Prototype
The first iteration of the prototype was a simple rectangular box. At the early stages of the design process, there was limited scope and clarity regarding the project requirements, prompting the development of a basic prototype to initiate discussions with stakeholders.
However, upon further refinement of the project specifications, it became evident that the initial rectangular design did not align with Sally's requirements, particularly in terms of size. The dimensions of the prototype were arbitrary and did not meet the specified 6x6x8cm dimensions. This discrepancy highlighted the need for clearer communication and adherence to design specifications from the outset of the project.
Additionally, the choice of PLA material for the prototype was based on availability, without consideration of alternative materials such as PETG, which offers superior properties for this application. The transition to PETG, as suggested by Ryan Jones, was later explored to address concerns related to material durability and performance.
One notable feature of the first iteration was the inclusion of a sliding mechanism for the open/close function. While this mechanism provided ease of access to the internal components, it posed challenges in terms of sealing effectiveness against moisture, dust, and water ingress. This limitation underscored the importance of robust sealing methods to ensure environmental protection and longevity of the electronic components housed within the enclosure.
Furthermore, the rectangular shape of the prototype presented challenges in airflow management, potentially leading to airflow turbulence or restrictions within the enclosure. While passive cooling solutions could be employed, such as ventilation holes, the design lacked provisions for active cooling mechanisms, which may be necessary to maintain optimal operating temperatures for sensitive electronics.
Despite these limitations, the rectangular design demonstrated efficiency in space utilization, offering maximum volume for component placement and organization. Moreover, its geometric simplicity facilitated ease of fabrication, reducing manufacturing complexity and costs associated with more intricate shapes.
Overall, the first iteration prototype served as a valuable learning experience, highlighting the importance of clear requirements definition, material selection, sealing methods, and considerations for airflow management in the design process. These insights informed subsequent iterations towards the development of a more refined and functional final prototype.

Second Iteration 
The second iteration resembled a pitched-roof house. This design idea appeared to be perfect until the size constraints were brought forth. Secondly, the pitched roof also limits and reduces the size of the already small dimensions of the specifications given by Sally: 6x6x8cm. 
This design had ventilation holes and forced cooling using a fan was considered. This fan would require a timer and trigger signal. This design had better airflow management as it promoted natural convection through the ventilation peaks at the peak without the need for additional cooling. The pitched roof enclosures blend seamlessly into outdoor environments, providing camouflage and enhancing the overall aesthetic appeal. However, concerns arose regarding the practicality of this design, particularly regarding size constraints and the need for additional cooling mechanisms as a fan would put added strain on the power supply and increase power losses.
Material used: Conductivity /insulation of material.
https://all3dp.com/2/pla-vs-abs-vs-petg-differences-compared/#google_vignette  
PLA, ABS and PETG are some of the most popular and readily available filaments out there. However, they are not equally suitable for every printing job as they have different qualities 
PLA (Polylactic Acid)
PLA is a widely used 3D printing material known for its ease of printing and low cost. Derived from renewable sources like corn starch, PLA is considered environmentally friendly, although it's more accurately described as industrially compostable rather than biodegradable. Its benefits include easy printing with no clogging issues, low print temperature, and affordability. However, PLA tends to deform or melt in high heat, making it unsuitable for heat-resistant applications. It's also brittle and less sturdy compared to materials like ABS or PETG, making it better suited for aesthetic uses rather than mechanical ones.
ABS (Acrylonitrile butadiene styrene)
ABS (Acrylonitrile Butadiene Styrene) is a durable thermoplastic commonly used in 3D printing, injection molding, and machining. While it offers increased durability and resistance to water compared to PLA, it can emit unpleasant fumes during printing. ABS is known for its strength, toughness, and heat resistance, making it suitable for everyday items. However, it may degrade over time when exposed to prolonged moisture, and it's prone to dust accumulation. Despite its drawbacks, ABS can be easily finished post-printing using acetone, making it suitable for aesthetic prints.
PETG (Polyethylene terephthalate glycol)
PETG is a popular choice for applications requiring food safety due to its excellent material properties. While it offers superior water resistance compared to PLA and ABS, printing PETG can be challenging due to its tendency to string during printing. PETG is known for its durability, impact resistance, and UV resistance, making it suitable for outdoor applications. However, it's prone to scratching and requires high printing temperatures, which contribute to stringing issues. Overall, PETG is widely regarded as a food-safe filament with excellent mechanical properties.

%----------------------------------------------------TABLE1-----------------------------------------------------------------------------------------------------------------------------------------------

Dual and Single enclosure considerations
In the evaluation of enclosure design options, a thorough analysis reveals nuanced advantages associated with both single and dual enclosure configurations. Single enclosures offer notable benefits such as efficient space utilization, simplified installation procedures, streamlined maintenance workflows, and potential cost savings due to component consolidation. However, they also pose challenges in thermal management, necessitating meticulous attention to airflow dynamics and heat dissipation strategies. Furthermore, comprehensive dustproofing measures are imperative to mitigate the risk of contamination and ensure long-term reliability.
Conversely, dual enclosure setups introduce unique advantages that contribute to the overall resilience and adaptability of the surveillance system. The modular nature of dual enclosures facilitates enhanced system flexibility, allowing for seamless component replacement, upgrades, or reconfigurations without disrupting the entire system. Moreover, the isolation of components within separate enclosures minimizes the risk of interference or contamination, ensuring optimal performance and reliability. The redundancy provided by dual enclosures offers an added layer of security, ensuring continuous operation even in the event of enclosure failure or malfunction.
Additionally, the customization potential inherent in dual enclosures enables tailored solutions to meet specific application requirements or environmental conditions. From optimized thermal management strategies to tailored dustproofing mechanisms, dual enclosures provide a versatile platform for fine-tuning system performance and resilience. By segregating components based on thermal characteristics, dual enclosures facilitate more effective heat distribution and dissipation, mitigating the risk of localized overheating and component degradation.
Furthermore, the independent sealing mechanisms of dual enclosures simplify dustproofing efforts, allowing for targeted measures to prevent dust ingress and contamination. By customizing dust filters and seals for each enclosure, the overall reliability and longevity of the surveillance system are enhanced, ensuring consistent performance even in harsh or dusty environments.

Shape Considerations
Each shape offers unique advantages and disadvantages, influencing factors such as space efficiency, aerodynamics, aesthetics, manufacturing complexity, and application-specific requirements.
Rectangular enclosures excel in efficient space utilization and ease of manufacture, making them suitable for accommodating complex electronic assemblies. However, they may lack optimal aerodynamics and aesthetic appeal compared to other shapes.
Cylindrical enclosures distribute stress more evenly and offer 360-degree access, making them robust and versatile for various applications. Yet, they may suffer from space inefficiency and manufacturing complexity.
Tapered or sloped enclosures prioritize improved aerodynamics and aesthetic appeal, optimizing space utilization while posing challenges in design complexity and manufacturing.


Final Design Prototype
The implementation of two separate electronic enclosure designs for the transmitter and receiver subsections was pivotal in addressing thermal concerns and meeting the distinct requirements of each subsystem. This separation not only improved temperature management but also allowed for increased airflow within each enclosure. Furthermore, considering that the transmitter and receiver subsystems are exposed to different environmental conditions and demand varied parameters for system longevity, this tailored approach ensured optimal performance and durability.
Transmitter Case
The final design for the mechanical cases for the transmitter subsection was chosen to be a rectangular box, carefully crafted to optimise functionality and environmental protection. It consisted of ventilation holes strategically placed to promote aeration and better airflow management, the design aimed to maintain optimal operating conditions for the enclosed components. Four vents were positioned at the base of the enclosure, each with a 5mm diameter and spaced 30mm apart, allowing air to enter from different locations. To prevent ingress of pests, debris, and dust, mesh screen filters covered these vents, while also minimizing the entry of water.
Four rectangular extrusions, situated 9mm above the base of the box, provided mounting or adhesive points for the Vero boards. This elevated placement reduced contact with any water that might enter through the base vents, ensuring the integrity of the components. Moreover, this design allowed for sufficient space for cool air to circulate around the boards.
To further enhance thermal management, aluminum heatsinks were strategically placed next to resistive and heat-prone components. The orientation of the heatsink fins facing vertically upwards facilitated the natural convection process, whereby warm air rises. This directed the heat away from the components and towards the top of the enclosure. This is because warm air has lower density than cold air hence it rises. 
Additionally, three vents on either side of the box were incorporated to facilitate the removal of warm air from within the enclosure. This prevented the accumulation of warm air and the formation of a low-pressure cell, ensuring efficient airflow throughout the enclosure.
To tightly seal the enclosure a lid-screw was used to minimise dust and water which was better than the sliding open/close mechanism used before.
For water and moisture management, a desiccant, silica gel was used to remove or absorb moisture from the electronic enclosure

Safety mechanisms
Mounting the rectangular enclosure is crucial for both practical functionality and ensuring safety and security. By firmly attaching the enclosure to a stable structure like a wall or pole, it greatly reduces the risk of unauthorized access to its internal components. This means it's much harder for people without permission to tamper with or steal the system. The act of mounting also acts as a strong deterrent against theft or vandalism, as it makes it clear that the system is secure and not easily removable. Additionally, securely mounting the enclosure ensures it can withstand external forces like wind or impact, making it more durable. It also helps the system comply with safety regulations, ensuring it meets industry standards. Adding features like tamper-resistant mechanisms further enhances security, making it even harder for unauthorized access. Overall, mounting the rectangular enclosure not only serves practical purposes but also plays a vital role in protecting the system and ensuring it functions effectively for a long time.
Mounting the enclosure was deemed a superior safety and security measure, particularly due to the elevated and often remote locations of Starlings' nests, where human access is infrequent. Opting for mounting over traditional locking mechanisms offers a practical solution, ensuring the system's protection without the complexities associated with locks. By firmly attaching the enclosure to a stable structure, such as a wall or pole, the system's integrity is effectively preserved, mitigating the risk of tampering or theft. This approach not only enhances security but also aligns with the practical constraints of the surveillance environment, where accessibility is limited.

In the ongoing development of the electronic enclosure, a transition has been made from a sliding open/close mechanism to a lid-screw configuration. This upgrade aims to provide a more robust closure, effectively minimizing the intrusion of dust and water, while also enhancing security for the enclosed electronics.
Additionally, to address water and moisture management within the enclosure, silica gel desiccant packs have been integrated into the design. These packs serve as an effective solution for moisture absorption, ensuring the internal environment remains dry and conducive to the optimal functioning of the electronic components. Furthermore, the potential addition of cotton alongside the silica gel desiccant packs can provide an extra layer of moisture protection, especially in environments with high humidity levels or where condensation may occur. Cotton's ability to absorb excess moisture and act as a buffer between electronic components and desiccant packs offers additional assurance against moisture infiltration.
Material of Choice
PETG was the material of choice for the enclosure. PETG offers UV resistance, ensuring the enclosure's structural integrity and optical clarity under prolonged sunlight exposure. Its weather resistance enables it to withstand extreme temperatures, both cold and hot, without compromising its mechanical properties. PETG's strength and load-bearing capacity make it suitable for housing electronic components securely. Importantly, PETG is non-toxic and does not emit harmful gases, aligning with your sustainability goals. Additionally, its recyclable nature promotes responsible material usage, further enhancing the project's sustainability. 

There were two enclosures designed for the transmitter subsection housing, one for the power module and the other for the sensing circuitry discussed in the other 2 sections of the report. The dual enclosure method was implemented based on the advantages discussed above. Moreover it allowed for modular design for each of the systems’ requirements. 
The power enclosure had all the design considerations discussed above: a lid screw, 4 vents at the bottom, 3 on either side of the box, 4 extrusions for Veroboard, and lastly 4 exterior extrusions with holes for mounting.
The sensing circuitry housing had the same design as the housing for power and had additional openings for the camera sensor, 5 PIR sensors and the 2 LDRs, so that surveillance can be achieved remotely while the housing is mounted. The housings are to be mounted very close to the nests which would be easy given where the nests are on campus. 
Receiver Case
The receiver module material was chosen to be ABS, since it is durable and can be used for consumer daily needs. ABS’ smooth texture and durability and impact resistance makes for the suitable material choice the hand-held receiver module. 
The enclosure was cut out to allow for the display of the LCD output with a clip side that was easily removable. This clip served as a great button access point as buttons to the receiver could be easily accessed without having to disassemble the enclosure. 
The enclosure also had the removable latch-sliding mechanism which also served as a port access and made for easy battery-changing as the battery would placed at the sliding mechanism side. The opening on the sliding mechanism side was big enough to allow hands to go through the  

