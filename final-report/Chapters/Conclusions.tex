% ----------------------------------------------------
\documentclass[class=report,11pt,crop=false]{standalone}
% Page geometry
\usepackage[a4paper,margin=20mm,top=25mm,bottom=25mm]{geometry}

% Font choice
\usepackage{lmodern}

% \usepackage{lipsum}

% Use IEEE bibliography style
\bibliographystyle{IEEEtran}

% Line spacing
\usepackage{setspace}
\setstretch{1.20}

% Ensure UTF8 encoding
\usepackage[utf8]{inputenc}

% Language standard (not too important)
\usepackage[english]{babel}

% Skip a line in between paragraphs
\usepackage{parskip}

% For the creation of dummy text
\usepackage{blindtext}

% Math
\usepackage{amsmath}

% Header & Footer stuff
\usepackage{fancyhdr}
\pagestyle{fancy}
\fancyhead{}
\fancyhead[R]{\nouppercase{\rightmark}}
\fancyfoot{}
\fancyfoot[C]{\thepage}
\renewcommand{\headrulewidth}{0.0pt}
\renewcommand{\footrulewidth}{0.0pt}
\setlength{\headheight}{13.6pt}

% Epigraphs
\usepackage{epigraph}
\setlength\epigraphrule{0pt}
\setlength{\epigraphwidth}{0.65\textwidth}

% Colour
\usepackage{color}
\usepackage[usenames,dvipsnames]{xcolor}

% Hyperlinks & References
\usepackage{hyperref}
\definecolor{linkColour}{RGB}{77,71,179}
\hypersetup{
    colorlinks=true,
    linkcolor=linkColour,
    filecolor=linkColour,
    urlcolor=linkColour,
    citecolor=linkColour,
}
\urlstyle{same}

% Automatically correct front-side quotes
\usepackage[autostyle=false, style=ukenglish]{csquotes}
\MakeOuterQuote{"}

% Graphics
\usepackage{graphicx}
\graphicspath{{Images/}{../Images/}}
\usepackage{makecell}
\usepackage{transparent}

% SI units
\usepackage{siunitx}

% Microtype goodness
\usepackage{microtype}

% Listings
\usepackage[T1]{fontenc}
\usepackage{listings}
\usepackage[scaled=0.8]{DejaVuSansMono}

% Custom colours for listings
\definecolor{backgroundColour}{RGB}{250,250,250}
\definecolor{commentColour}{RGB}{73, 175, 102}
\definecolor{identifierColour}{RGB}{196, 19, 66}
\definecolor{stringColour}{RGB}{252, 156, 30}
\definecolor{keywordColour}{RGB}{50, 38, 224}
\definecolor{lineNumbersColour}{RGB}{127,127,127}
\lstset{
  language=Matlab,
  captionpos=b,
  aboveskip=15pt,belowskip=10pt,
  backgroundcolor=\color{backgroundColour},
  basicstyle=\ttfamily,%\footnotesize,        % the size of the fonts that are used for the code
  breakatwhitespace=false,         % sets if automatic breaks should only happen at whitespace
  breaklines=true,                 % sets automatic line breaking
  postbreak=\mbox{\textcolor{red}{$\hookrightarrow$}\space},
  commentstyle=\color{commentColour},    % comment style
  identifierstyle=\color{identifierColour},
  stringstyle=\color{stringColour},
   keywordstyle=\color{keywordColour},       % keyword style
  %escapeinside={\%*}{*)},          % if you want to add LaTeX within your code
  extendedchars=true,              % lets you use non-ASCII characters; for 8-bits encodings only, does not work with UTF-8
  frame=single,	                   % adds a frame around the code
  keepspaces=true,                 % keeps spaces in text, useful for keeping indentation of code (possibly needs columns=flexible)
  morekeywords={*,...},            % if you want to add more keywords to the set
  numbers=left,                    % where to put the line-numbers; possible values are (none, left, right)
  numbersep=5pt,                   % how far the line-numbers are from the code
  numberstyle=\tiny\color{lineNumbersColour}, % the style that is used for the line-numbers
  rulecolor=\color{black},         % if not set, the frame-color may be changed on line-breaks within not-black text (e.g. comments (green here))
  showspaces=false,                % show spaces everywhere adding particular underscores; it overrides 'showstringspaces'
  showstringspaces=false,          % underline spaces within strings only
  showtabs=false,                  % show tabs within strings adding particular underscores
  stepnumber=1,                    % the step between two line-numbers. If it's 1, each line will be numbered
  tabsize=2,	                   % sets default tabsize to 2 spaces
  %title=\lstname                   % show the filename of files included with \lstinputlisting; also try caption instead of title
}

% Caption stuff
\usepackage[hypcap=true, justification=centering]{caption}
\usepackage{subcaption}

% Glossary package
% \usepackage[acronym]{glossaries}
\usepackage{glossaries-extra}
\setabbreviationstyle[acronym]{long-short}

% For Proofs & Theorems
\usepackage{amsthm}

% Maths symbols
\usepackage{amssymb}
\usepackage{mathrsfs}
\usepackage{mathtools}

% For algorithms
\usepackage[]{algorithm2e}

% Spacing stuff
\setlength{\abovecaptionskip}{5pt plus 3pt minus 2pt}
\setlength{\belowcaptionskip}{5pt plus 3pt minus 2pt}
\setlength{\textfloatsep}{10pt plus 3pt minus 2pt}
\setlength{\intextsep}{15pt plus 3pt minus 2pt}

% For aligning footnotes at bottom of page, instead of hugging text
\usepackage[bottom]{footmisc}

% Add LoF, Bib, etc. to ToC
\usepackage[nottoc]{tocbibind}

% SI
\usepackage{siunitx}

% For removing some whitespace in Chapter headings etc
\usepackage{etoolbox}
\makeatletter
\patchcmd{\@makechapterhead}{\vspace*{50\p@}}{\vspace*{-10pt}}{}{}%
\patchcmd{\@makeschapterhead}{\vspace*{50\p@}}{\vspace*{-10pt}}{}{}%
\makeatother

% Tables
\usepackage{multirow}
\usepackage{longtable}
\usepackage{tabularx}
\makenoidxglossaries

\newacronym{radar}{RADAR}{Radio Detection and Ranging}
\begin{document}
\chapter{Conclusion \label{ch:conclusions}}

The purpose of this project was to develop a nest monitoring system tailor built for the nest environments of the Red-Winged Starling situated around UCT. The proposed solution was to develop a receiver and transmitter that wirelessly connect to each other for data transmission. The ESP32-CAM was the main hub for the transmitter module. An OV2640 camera was positioned along with motion sensors, reducing the chance of missing important events in the nest. This information would then be transmitted to the receiver upon the users' request. 

This report began with a short introduction on the project. This included looking at the background information on The FitzPatrick Institute of African Ornithology's Red-winged Starling study. The problem statement was also defined, based on the problems researchers endure when studying the Red-winged Starlings. The scope and limitations were addressed, which included the financial constraints and limited available facilities to order from. Finally, the report outline was also discussed. 

The literature review was followed in Chapter \ref{ch:literature}. This was an exploration of current solutions relating to the project, different computing hardware used, motion detection, Image/video capturing hardware, data transmission techniques, power demands, before finally looking at the challenges of deploying camera traps in urban settings. 

The bulk of the work for this project followed next. Chapter \ref{ch:power} documented the full design process of the power supply for both the receiver and transmitter/camera module, while taking into account the power demands of the components and space constraints laid out by the stakeholders. 

Chapter \ref{ch:hardware} detailed the entire design process of the camera/transmitter and receiver hardware, with all design decisions accounted for, ending off with a tested and working final design. 

The firmware used on the hardware developed was discussed in Chapter \ref{ch:firmware}, where all decisions made were discussed. The final firmware solution was then rigourously tested to ensure the interface was reliable. 

Finally, Chapter \ref{ch:mechanical} discussed the design and testing of the weatherproof casing for the camera/transmitter and handheld box for the receiver. 

In summary, the project achieved the goals that were set out, by designing and eploying a working nest monitoring system suited to the nesting environments of Red-winged Starlings at the University of Cape Town. This design will hopefully allow the researcher to capture useful images of the birds, and reduces the risk of missing important milestones in the life cycle of the Red-winged Starlings. 

\chapter{Recommendations}


% ----------------------------------------------------
\ifstandalone
\bibliography{../Bibliography/References.bib}
\printnoidxglossary[type=\acronymtype,nonumberlist]
\fi
\end{document}
% ----------------------------------------------------