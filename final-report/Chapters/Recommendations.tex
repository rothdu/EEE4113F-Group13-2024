% ----------------------------------------------------
% Recommendations
% ----------------------------------------------------
\documentclass[class=report,11pt,crop=false]{standalone}
\input{../Style/ChapterStyle.tex}
\input{../FrontMatter/Glossary.tex}
\begin{document}
\chapter{Recommendations \label{ch:recommendations}}

When designing this system, there were some features that could be added to make the overall design more robust. 

One major problem is that the current hardware is unable to take night vision shots. A better camera, without an IR filter, could be implemented. This would allow the camera to pick up the IR light being shone on the nest during dark conditions. The current design has a camera with a permanent IR filter. The IR light cannot be seen in night shots. Another feature that could be added to the camera system is video capture. Currently, the system can only take photos. Videos would require more attention to the storage rate and better management of the power consumption. 

With regards to the casing, a Glass Protection for Camera Nozzle could be added. This can provide an extra layer of protection against dust, debris, and weather elements while still allowing clear visibility for the camera. This helps to safeguard the camera lens and maintain optimal image quality over time.  Angling the side vents can help to deflect dust, water, and other contaminants away from the interior of the enclosure, reducing the likelihood of ingress. This improves the effectiveness of the ventilation system while maintaining protection against environmental factors. Installing mesh screen filters on the side vents further enhances dust and debris protection by preventing larger particles from entering the enclosure. The mesh screens allow for airflow while effectively blocking unwanted materials, ensuring clean and efficient ventilation.Incorporating sealed cable glands for cable entry points ensures a watertight seal around cables, preventing moisture ingress into the enclosure. This enhances the overall weather resistance and protects internal electronics from potential water damage. Integrate shock absorption materials or mechanisms within the enclosure design to cushion electronic components against impact and vibration. This helps to mitigate the risk of damage from sudden shocks or movement, prolonging the lifespan of sensitive equipment.

The current solution is a great starting point. These recommendations would help enhance it. 


% ----------------------------------------------------
\ifstandalone
\bibliography{../Bibliography/References.bib}
\printnoidxglossary[type=\acronymtype,nonumberlist]
\fi
\end{document}
% ----------------------------------------------------
