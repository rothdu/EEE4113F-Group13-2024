% ----------------------------------------------------
% Introduction
% ----------------------------------------------------
\documentclass[class=report,11pt,crop=false]{standalone}
\input{../Style/ChapterStyle.tex}
\input{../FrontMatter/Glossary.tex}
\begin{document}
\chapter{Introduction}
% ----------------------------------------------------



\section{Background}
The user wishes to observe the niche behaviours of Red-winged Starlings during the critical phases of their nesting period, which include hatching, predation, and fledging. The user is also interested in viewing activities around the nest such as human foot traffic and predators that visit the nest. Currently, most of these important details are missed, which often leads to disappointment. To not miss any more important details, the transmitter module that will be situated on the nest will be designed to trigger the camera if motion is detected in and around the nest. Depending on the amount of motion detected, the transmitter will capture still images and/or videos, providing researchers with valuable information of the Red-winged Starlings’ nesting process. The captured data will be stored onboard the transmitter module and transmitted to a handheld receiver located on the ground. This remote data capture prevents disturbance and avoids causing distress to the Red-winged Starlings. 

The solution requires a reliable power source to keep the transmitter active over a few days. A weatherproof case is also designed to protect the camera and all electronic components from damage. 

\section{Objectives}


\section{System Requirements}


\section{Scope \& Limitations}


\section{Report Outline}


% ----------------------------------------------------
\ifstandalone
\bibliography{../Bibliography/References.bib}
\printnoidxglossary[type=\acronymtype,nonumberlist]
\fi
\end{document}
% ----------------------------------------------------