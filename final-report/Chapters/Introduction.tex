% ----------------------------------------------------
% Introduction
% ----------------------------------------------------
\documentclass[class=report,11pt,crop=false]{standalone}
% Page geometry
\usepackage[a4paper,margin=20mm,top=25mm,bottom=25mm]{geometry}

% Font choice
\usepackage{lmodern}

% \usepackage{lipsum}

% Use IEEE bibliography style
\bibliographystyle{IEEEtran}

% Line spacing
\usepackage{setspace}
\setstretch{1.20}

% Ensure UTF8 encoding
\usepackage[utf8]{inputenc}

% Language standard (not too important)
\usepackage[english]{babel}

% Skip a line in between paragraphs
\usepackage{parskip}

% For the creation of dummy text
\usepackage{blindtext}

% Math
\usepackage{amsmath}

% Header & Footer stuff
\usepackage{fancyhdr}
\pagestyle{fancy}
\fancyhead{}
\fancyhead[R]{\nouppercase{\rightmark}}
\fancyfoot{}
\fancyfoot[C]{\thepage}
\renewcommand{\headrulewidth}{0.0pt}
\renewcommand{\footrulewidth}{0.0pt}
\setlength{\headheight}{13.6pt}

% Epigraphs
\usepackage{epigraph}
\setlength\epigraphrule{0pt}
\setlength{\epigraphwidth}{0.65\textwidth}

% Colour
\usepackage{color}
\usepackage[usenames,dvipsnames]{xcolor}

% Hyperlinks & References
\usepackage{hyperref}
\definecolor{linkColour}{RGB}{77,71,179}
\hypersetup{
    colorlinks=true,
    linkcolor=linkColour,
    filecolor=linkColour,
    urlcolor=linkColour,
    citecolor=linkColour,
}
\urlstyle{same}

% Automatically correct front-side quotes
\usepackage[autostyle=false, style=ukenglish]{csquotes}
\MakeOuterQuote{"}

% Graphics
\usepackage{graphicx}
\graphicspath{{Images/}{../Images/}}
\usepackage{makecell}
\usepackage{transparent}

% SI units
\usepackage{siunitx}

% Microtype goodness
\usepackage{microtype}

% Listings
\usepackage[T1]{fontenc}
\usepackage{listings}
\usepackage[scaled=0.8]{DejaVuSansMono}

% Custom colours for listings
\definecolor{backgroundColour}{RGB}{250,250,250}
\definecolor{commentColour}{RGB}{73, 175, 102}
\definecolor{identifierColour}{RGB}{196, 19, 66}
\definecolor{stringColour}{RGB}{252, 156, 30}
\definecolor{keywordColour}{RGB}{50, 38, 224}
\definecolor{lineNumbersColour}{RGB}{127,127,127}
\lstset{
  language=Matlab,
  captionpos=b,
  aboveskip=15pt,belowskip=10pt,
  backgroundcolor=\color{backgroundColour},
  basicstyle=\ttfamily,%\footnotesize,        % the size of the fonts that are used for the code
  breakatwhitespace=false,         % sets if automatic breaks should only happen at whitespace
  breaklines=true,                 % sets automatic line breaking
  postbreak=\mbox{\textcolor{red}{$\hookrightarrow$}\space},
  commentstyle=\color{commentColour},    % comment style
  identifierstyle=\color{identifierColour},
  stringstyle=\color{stringColour},
   keywordstyle=\color{keywordColour},       % keyword style
  %escapeinside={\%*}{*)},          % if you want to add LaTeX within your code
  extendedchars=true,              % lets you use non-ASCII characters; for 8-bits encodings only, does not work with UTF-8
  frame=single,	                   % adds a frame around the code
  keepspaces=true,                 % keeps spaces in text, useful for keeping indentation of code (possibly needs columns=flexible)
  morekeywords={*,...},            % if you want to add more keywords to the set
  numbers=left,                    % where to put the line-numbers; possible values are (none, left, right)
  numbersep=5pt,                   % how far the line-numbers are from the code
  numberstyle=\tiny\color{lineNumbersColour}, % the style that is used for the line-numbers
  rulecolor=\color{black},         % if not set, the frame-color may be changed on line-breaks within not-black text (e.g. comments (green here))
  showspaces=false,                % show spaces everywhere adding particular underscores; it overrides 'showstringspaces'
  showstringspaces=false,          % underline spaces within strings only
  showtabs=false,                  % show tabs within strings adding particular underscores
  stepnumber=1,                    % the step between two line-numbers. If it's 1, each line will be numbered
  tabsize=2,	                   % sets default tabsize to 2 spaces
  %title=\lstname                   % show the filename of files included with \lstinputlisting; also try caption instead of title
}

% Caption stuff
\usepackage[hypcap=true, justification=centering]{caption}
\usepackage{subcaption}

% Glossary package
% \usepackage[acronym]{glossaries}
\usepackage{glossaries-extra}
\setabbreviationstyle[acronym]{long-short}

% For Proofs & Theorems
\usepackage{amsthm}

% Maths symbols
\usepackage{amssymb}
\usepackage{mathrsfs}
\usepackage{mathtools}

% For algorithms
\usepackage[]{algorithm2e}

% Spacing stuff
\setlength{\abovecaptionskip}{5pt plus 3pt minus 2pt}
\setlength{\belowcaptionskip}{5pt plus 3pt minus 2pt}
\setlength{\textfloatsep}{10pt plus 3pt minus 2pt}
\setlength{\intextsep}{15pt plus 3pt minus 2pt}

% For aligning footnotes at bottom of page, instead of hugging text
\usepackage[bottom]{footmisc}

% Add LoF, Bib, etc. to ToC
\usepackage[nottoc]{tocbibind}

% SI
\usepackage{siunitx}

% For removing some whitespace in Chapter headings etc
\usepackage{etoolbox}
\makeatletter
\patchcmd{\@makechapterhead}{\vspace*{50\p@}}{\vspace*{-10pt}}{}{}%
\patchcmd{\@makeschapterhead}{\vspace*{50\p@}}{\vspace*{-10pt}}{}{}%
\makeatother

% Tables
\usepackage{multirow}
\usepackage{longtable}
\usepackage{tabularx}
\makenoidxglossaries

\newacronym{radar}{RADAR}{Radio Detection and Ranging}
\begin{document}
\chapter{Introduction \label{ch:introduction}}
% ----------------------------------------------------



\section{Background}
\section{Background}
The FitzPatrick Institute of African Ornithology falls within the Department of Biological Sciences at the University of Cape Town (UCT) in South Africa. It is a non-profit organisation that established it’s affiliation with UCT in September 1959. The main aim of FitzPatrick is to deepen understanding of avian life and to establish a permanent repository for ornithological data and materials crucial for research. They serve as pioneers and are one of the leading institutes when it comes to the studies of birds in the Southern Hemisphere.
\newline

The FitzPatrick Institute is also committed to training the next generation of ornithologists and conservationists. It offers undergraduate and postgraduate programs in ornithology, providing students with hands-on research experience and mentorship from leading experts in the field.
\newline

Overall, the FitzPatrick Institute of African Ornithology stands as a beacon of excellence in avian research and conservation, contributing valuable insights to our understanding of bird biology and ecology while working tirelessly to ensure the continued survival of Africa's avian biodiversity.
\newline

In 2017, the FitzPatrick initiated the Red-winged Starling Urban Ecology Project. The Red-winged starling, scientifically termed Onychognathus morio, is a notable bird within the Sturnidae family, endemic to eastern Africa from Ethiopia to South Africa's Cape region. For this specific project, we worked with Sally, who does her research on these starling birds.
\newline

Her main problems along her researching career were not being able to understand exactly what is going on in these nests of the Starling birds. Hence, her need for a camera trap device so that she can have more accurate and efficient data.


%%\section{Objectives}


%%\section{System Requirements}


\section{Scope \& Limitations}
The project focused on developing a remote surveillance system for the Starlings bird population at the University of Cape Town (UCT), which was divided into four main subsections: Power, Hardware, Firmware, and Mechanical Casing. the project aimed to develop a technological design to assist in wildlife data collection and conservation. This endeavour involved a collaborative effort including various engineering disciplines and leveraging the vast knowledge and research capabilities of the Fitz Patrick Institute.
\newline
Over 7 weeks, the project encompassed stakeholder engagement for gathering requirements, the design process, iterative improvements, and thorough testing. Material procurement was conducted from limited electronics online stores, with shipping days taking long and physical visits to outlets like BUCO for items not available online. A budget of R2000 was allocated to ensure a cost-effective solution.
\newline

In the subsequent chapters, the design processes of each submodule will be explained, showcasing diverse expertise and innovative solutions tailored to the unique requirements of monitoring the Starlings bird population. 

\section{Report Outline}
The report will proceed in a structured manner to provide a comprehensive analysis and solution to the identified problem. Beginning with the Introduction, the report will establish the context and significance of the problem, setting the stage for subsequent chapters. Following this, the Problem Analysis section will delve into a detailed examination of the identified issue, analysing its implications and significance within the relevant context.
\newline

Subsequently, a Literature Review will be conducted to draw upon existing research and scholarly work related to the problem statement. This review will serve to establish a theoretical framework and deepen understanding of the topic. Moving forward, chapters dedicated to Power, Hardware, Mechanical, and Firmware will follow. Each chapter will offer an overview of its respective aspect, analysing requirements, considerations, and design strategies pertinent to the solution.
\newline

The Conclusions section will summarise key findings and insights drawn from the analysis, offering recommendations for future actions and areas of further research. References will be provided to acknowledge and cite sources utilized throughout the report.
\newline
Finally, the report includes two appendices. Appendix A, the main appendix, contain supplementary information, data, or detailed analyses to support the main body of the report. Appendix B, consists of the GAs completed by each member of Group 13.


% ----------------------------------------------------
\ifstandalone
\bibliography{../Bibliography/References.bib}
\printnoidxglossary[type=\acronymtype,nonumberlist]
\fi
\end{document}
% ----------------------------------------------------
