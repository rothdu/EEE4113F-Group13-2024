% ----------------------------------------------------
% Introduction
% ----------------------------------------------------
\documentclass[class=report,11pt,crop=false]{standalone}
\input{../Style/ChapterStyle.tex}
\input{../FrontMatter/Glossary.tex}
\begin{document}
\chapter{Introduction \label{ch:introduction}}
% ----------------------------------------------------



\section{Background}
\section{Background}
The FitzPatrick Institute of African Ornithology falls within the Department of Biological Sciences at the University of Cape Town (UCT) in South Africa. It is a non-profit organisation that established it’s affiliation with UCT in September 1959. The main aim of FitzPatrick is to deepen understanding of avian life and to establish a permanent repository for ornithological data and materials crucial for research. They serve as pioneers and are one of the leading institutes when it comes to the studies of birds in the Southern Hemisphere.
\newline

The FitzPatrick Institute is also committed to training the next generation of ornithologists and conservationists. It offers undergraduate and postgraduate programs in ornithology, providing students with hands-on research experience and mentorship from leading experts in the field.
\newline

Overall, the FitzPatrick Institute of African Ornithology stands as a beacon of excellence in avian research and conservation, contributing valuable insights to our understanding of bird biology and ecology while working tirelessly to ensure the continued survival of Africa's avian biodiversity.
\newline

In 2017, the FitzPatrick initiated the Red-winged Starling Urban Ecology Project. The Red-winged starling, scientifically termed Onychognathus morio, is a notable bird within the Sturnidae family, endemic to eastern Africa from Ethiopia to South Africa's Cape region. For this specific project, we worked with Sally, who does her research on these starling birds.
\newline

Her main problems along her researching career were not being able to understand exactly what is going on in these nests of the Starling birds. Hence, her need for a camera trap device so that she can have more accurate and efficient data.


\section{Objectives}


\section{System Requirements}


\section{Scope \& Limitations}


\section{Report Outline}


% ----------------------------------------------------
\ifstandalone
\bibliography{../Bibliography/References.bib}
\printnoidxglossary[type=\acronymtype,nonumberlist]
\fi
\end{document}
% ----------------------------------------------------
