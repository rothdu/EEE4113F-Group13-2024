\documentclass[class=report,11pt,crop=false]{standalone}
% Page geometry
\usepackage[a4paper,margin=20mm,top=25mm,bottom=25mm]{geometry}

% Font choice
\usepackage{lmodern}

% \usepackage{lipsum}

% Use IEEE bibliography style
\bibliographystyle{IEEEtran}

% Line spacing
\usepackage{setspace}
\setstretch{1.20}

% Ensure UTF8 encoding
\usepackage[utf8]{inputenc}

% Language standard (not too important)
\usepackage[english]{babel}

% Skip a line in between paragraphs
\usepackage{parskip}

% For the creation of dummy text
\usepackage{blindtext}

% Math
\usepackage{amsmath}

% Header & Footer stuff
\usepackage{fancyhdr}
\pagestyle{fancy}
\fancyhead{}
\fancyhead[R]{\nouppercase{\rightmark}}
\fancyfoot{}
\fancyfoot[C]{\thepage}
\renewcommand{\headrulewidth}{0.0pt}
\renewcommand{\footrulewidth}{0.0pt}
\setlength{\headheight}{13.6pt}

% Epigraphs
\usepackage{epigraph}
\setlength\epigraphrule{0pt}
\setlength{\epigraphwidth}{0.65\textwidth}

% Colour
\usepackage{color}
\usepackage[usenames,dvipsnames]{xcolor}

% Hyperlinks & References
\usepackage{hyperref}
\definecolor{linkColour}{RGB}{77,71,179}
\hypersetup{
    colorlinks=true,
    linkcolor=linkColour,
    filecolor=linkColour,
    urlcolor=linkColour,
    citecolor=linkColour,
}
\urlstyle{same}

% Automatically correct front-side quotes
\usepackage[autostyle=false, style=ukenglish]{csquotes}
\MakeOuterQuote{"}

% Graphics
\usepackage{graphicx}
\graphicspath{{Images/}{../Images/}}
\usepackage{makecell}
\usepackage{transparent}

% SI units
\usepackage{siunitx}

% Microtype goodness
\usepackage{microtype}

% Listings
\usepackage[T1]{fontenc}
\usepackage{listings}
\usepackage[scaled=0.8]{DejaVuSansMono}

% Custom colours for listings
\definecolor{backgroundColour}{RGB}{250,250,250}
\definecolor{commentColour}{RGB}{73, 175, 102}
\definecolor{identifierColour}{RGB}{196, 19, 66}
\definecolor{stringColour}{RGB}{252, 156, 30}
\definecolor{keywordColour}{RGB}{50, 38, 224}
\definecolor{lineNumbersColour}{RGB}{127,127,127}
\lstset{
  language=Matlab,
  captionpos=b,
  aboveskip=15pt,belowskip=10pt,
  backgroundcolor=\color{backgroundColour},
  basicstyle=\ttfamily,%\footnotesize,        % the size of the fonts that are used for the code
  breakatwhitespace=false,         % sets if automatic breaks should only happen at whitespace
  breaklines=true,                 % sets automatic line breaking
  postbreak=\mbox{\textcolor{red}{$\hookrightarrow$}\space},
  commentstyle=\color{commentColour},    % comment style
  identifierstyle=\color{identifierColour},
  stringstyle=\color{stringColour},
   keywordstyle=\color{keywordColour},       % keyword style
  %escapeinside={\%*}{*)},          % if you want to add LaTeX within your code
  extendedchars=true,              % lets you use non-ASCII characters; for 8-bits encodings only, does not work with UTF-8
  frame=single,	                   % adds a frame around the code
  keepspaces=true,                 % keeps spaces in text, useful for keeping indentation of code (possibly needs columns=flexible)
  morekeywords={*,...},            % if you want to add more keywords to the set
  numbers=left,                    % where to put the line-numbers; possible values are (none, left, right)
  numbersep=5pt,                   % how far the line-numbers are from the code
  numberstyle=\tiny\color{lineNumbersColour}, % the style that is used for the line-numbers
  rulecolor=\color{black},         % if not set, the frame-color may be changed on line-breaks within not-black text (e.g. comments (green here))
  showspaces=false,                % show spaces everywhere adding particular underscores; it overrides 'showstringspaces'
  showstringspaces=false,          % underline spaces within strings only
  showtabs=false,                  % show tabs within strings adding particular underscores
  stepnumber=1,                    % the step between two line-numbers. If it's 1, each line will be numbered
  tabsize=2,	                   % sets default tabsize to 2 spaces
  %title=\lstname                   % show the filename of files included with \lstinputlisting; also try caption instead of title
}

% Caption stuff
\usepackage[hypcap=true, justification=centering]{caption}
\usepackage{subcaption}

% Glossary package
% \usepackage[acronym]{glossaries}
\usepackage{glossaries-extra}
\setabbreviationstyle[acronym]{long-short}

% For Proofs & Theorems
\usepackage{amsthm}

% Maths symbols
\usepackage{amssymb}
\usepackage{mathrsfs}
\usepackage{mathtools}

% For algorithms
\usepackage[]{algorithm2e}

% Spacing stuff
\setlength{\abovecaptionskip}{5pt plus 3pt minus 2pt}
\setlength{\belowcaptionskip}{5pt plus 3pt minus 2pt}
\setlength{\textfloatsep}{10pt plus 3pt minus 2pt}
\setlength{\intextsep}{15pt plus 3pt minus 2pt}

% For aligning footnotes at bottom of page, instead of hugging text
\usepackage[bottom]{footmisc}

% Add LoF, Bib, etc. to ToC
\usepackage[nottoc]{tocbibind}

% SI
\usepackage{siunitx}

% For removing some whitespace in Chapter headings etc
\usepackage{etoolbox}
\makeatletter
\patchcmd{\@makechapterhead}{\vspace*{50\p@}}{\vspace*{-10pt}}{}{}%
\patchcmd{\@makeschapterhead}{\vspace*{50\p@}}{\vspace*{-10pt}}{}{}%
\makeatother

% Tables
\usepackage{multirow}
\usepackage{longtable}
\usepackage{tabularx}
\makenoidxglossaries

\newacronym{radar}{RADAR}{Radio Detection and Ranging}
\begin{document}
\ifstandalone
\tableofcontents
\fi
%------------------------------------------------------------

% \appendix{Evidence of GAs met}

\chapter{Evidence of GAs Met}

\section{PRMRUV001}

\centering
\begin{tabularx}{\textwidth}{|p{0.07\textwidth} p{0.2\textwidth} X|}

    \hline
    \textbf{GA} & \textbf{GA Description} & \textbf{Area GA is met} \\ \hline

    GA3 & Engineering Design & Chapter \ref{ch:power} (Power Chapter) details the development of the power supply for the Red-Winged Starling Nest Monitoring project. After engagement with project stakeholders, the requirements and specifications of the subsystem were anaylsed. A number of high-level design decisions were considered and detailed, followed by extensive testing of the subsystem on the project hardware. Finally, a number of acceptance test procedures were developed and performed. \\ \hline

    GA7 & Sustainability and Impact of Engineering Activity & The Red-Winged Starling Nest Monitoring works towards the sustainability of engineering activity by developing a tool for research in wildlife conservation. In addition, D-school lessons were attended where we actively participated in ethical design practices. \\ \hline

    GA8 & Individual, Team and Multidiscipliniary Working & The overall project encompassed extensive teamwork, including a collective analysis of the needs of the project stakeholders and a joint literature review. Much of this communication is viewable on the group's Microsoft Teams chat. Within the framework of the group project, individual work was done in the development of the power system subsystem, this is evident in the power submodule section of the final report. The group members encompassed a multidiscipliniary engineering team, with specialisations in Electrical Engineer, Electrical and Computer Engineering, and Mechatronics. The project also involved extensive engagement with ornothological researchers, who work outside of the engineering field. \\ \hline

    GA10 & Engineering Professionalism & Throughout the project, professional relationships and communication were maintained with the other group members, project stakeholders and project staff. All deliverables were presented on time. A schedule was made and stuck to in terms of being on time. \\ \hline


\end{tabularx}
\raggedright







\section{PLLTHI032}

\centering
\begin{tabularx}{\textwidth}{|p{0.25\textwidth} X|}

    \hline
    \textbf{Camera/Transmitter (CT) \& Receiver (R) Hardware (PLLTHI032)}  & \textbf{Area GA is met} \\ \hline

    GA3: Engineering Design & Chapter \ref{ch:hardware} discusses the development of the transmitter/camera and receiver modules for the Red-Winged Starling Nest Monitoring project. After speaking to the stakeholders, the user requirements, functional requirements and specifications were created and analysed, shown in section \ref{sc: HW_R_S}. Thereafter, the entire design process was documented and shown in section \ref{sc: HW_DP}. This was followed by a proposed final design and development of support circuitry, shown in section \ref{sc: HW_FD}. The power consumption was then measured in different operating modes, shown in section \ref{sc: HW_PC}. Finally, a number of acceptance test procedures were developed and performed, shown in section \ref{sc: HW_TR}. \\ \hline

    GA7: Sustainability and Impact of Engineering Activity & The Red-Winged Starling Nest Monitoring project works towards sustainability of engineering activity by creating a product for wildlife research and conservation.  In addition, the group attended D-School to develop skills on design approaches. \\ \hline

    GA8:Individual, Team and Multidiscipliniary Working & Throughout the project, teamwork was key in making sure the final product worked together harmoniously. This involved creating a joint literature review, along with communicating weekly on Microsoft Teams. Extensive discussions were conducted, with the main objective to find the perfect solution to meet the stakeholder requirements, as well as dividing the work into equal subsystems. Thereafter, individual work was conducted to develop the hardware subsystem of the camera/transmitter and receiver on my end. The members that fromed group 13 were of different engineering backgrounds such as Electrical Engineering, Electrical and Computer Engineering, and Mechatronics. Extensive research was conducted with ornothological researchers who are not involved in the engineering space. \\ \hline

    GA10: Engineering Professionalism & All submissions and activities were met including the final report and presentation. Professional relationships have also been maintained between colleagues. \\ \hline


\end{tabularx}
\raggedright


\section{DGMROB001}

\centering
\begin{tabularx}{\textwidth}{|p{0.25\textwidth} X|}

    \hline
    \textbf{GA}  & \textbf{Area GA is met} \\ \hline

    GA3: Engineering Design & Chapter \ref{ch:firmware} details the development of firmware for the Red-Winged Starling Nest Monitoring project. After engagement with project stakeholders, the requirements and specifications of the subsystem were anaylsed. A number of high-level design decisions were considered and detailed in section \ref{s:firmware-design-decisions}, followed by extensive testing of the subsystem on the project hardware. Finally, a number of acceptance test procedures were developed and performed, discussed in section \ref{s:firmware-atps}. The source code for the firmware is available in the \href{https://github.com/rothdu/EEE4113F-Group13-2024}{project repository} on GitHub. \\ \hline

    GA7: Sustainability and Impact of Engineering Activity & The Red-Winged Straling Nest Monitoring works towards sustainability of engineering activity by developing a tool for research in wildlife conservation. Activities at the UCT D-School worked towards engaging with stakeholders on this issue. A broad overview of the project is given in Chapters \ref{ch:introduction} and \ref{ch:problem-analysis}. The firmware developed for this purpose is discussed in \ref{ch:firmware}.\\ \hline

    GA8: Individual, Team and Multidiscipliniary Working & The overall project encompassed extensive teamwork, including a collective analysis of the needs of the project stakeholders and a joint literature review. Minutes from meetings and intra-team communications are viewable on the project group's Microsoft Teams chat. Within the framework of the group project, individual work was done in the development of the firmware subsystem. The group members encompassed a multidiscipliniary engineering team, with specialisations in Electrical Engineering, Electrical and Computer Engineering, and Mechatronics. The project also involved extensive engagement with ornothological researchers, who work outside of the engineering field. \\ \hline

    GA10: Engineering Professionalism & Throughout the project, professional relationships and communication were maintained with the other group members, project stakeholders and project staff. All deliverables were presented on time. \\ \hline


\end{tabularx}
\raggedright

\section{MRBMAT004}

\centering
\begin{tabularx}{\textwidth}{|p{0.07\textwidth} p{0.2\textwidth} X|}

    \hline
    \textbf{GA} & \textbf{GA Description} & \textbf{Area GA is met} \\ \hline

    GA3 & Engineering Design & Chapter  \ref{ch:mechanical} (mechanical casing and housing) details the development of the housing and the design decisions taken for the Red-Winged Starling Nest Monitoring project. After thorough research and engagement with project stakeholders, the user requirements, functional requirements, and specifications were created and analysed, as shown in the Mechanical casing and housing section.   The source code for the firmware is available in the \href{https://github.com/rothdu/EEE4113F-Group13-2024}{project repository} on GitHub.   \\ \hline

    GA7 & Sustainability and Impact of Engineering Activity & The Red-Winged Starling Nest Monitoring project works towards the sustainability of engineering activity by developing a tool for research in wildlife conservation.  In addition, the group attended D-School to develop skills in design approaches. PETG was the chosen material based on its durability and its recyclable nature. \\ \hline

    GA8 & Individual, Team, and Multidisciplinary Working & Throughout the project, teamwork was key in making sure the final product was integrated successfully from the individual piece sections. This involved creating a joint literature review, along with communicating weekly on Microsoft Teams. Extensive discussions were conducted, with the main objective of finding the perfect solution to meet the stakeholder requirements, as well as dividing the work into equal subsystems. Thereafter, individual work was conducted to the 3 modular cases capable of housing and meeting the requirements of their environment and components weight. The group members encompassed a multidisciplinary engineering team, with specialisations in Electrical engineering, Electrical and Computer Engineering, and Mechatronics. The project also involved extensive engagement with ornithological researchers, who work outside of the engineering field. \\ \hline

    GA10 & Engineering Professionalism & Throughout the project, professional relationships and communication were maintained with the other group members, project stakeholders and project staff. All submission activities were submitted on time including final report and presentation.  \\ \hline


\end{tabularx}
\raggedright




%------------------------------------------------------------
\ifstandalone
\bibliography{../Bibliography/References.bib}
\printnoidxglossary[type=\acronymtype,nonumberlist]
\fi
\end{document}




%------------------------------------------------------------
\ifstandalone
\bibliography{../Bibliography/References.bib}
\printnoidxglossary[type=\acronymtype,nonumberlist]
\fi
\end{document}
