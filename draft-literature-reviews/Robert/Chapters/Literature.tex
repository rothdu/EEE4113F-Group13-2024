% ----------------------------------------------------
% Literature Review
% ----------------------------------------------------
\documentclass[class=report,11pt,crop=false]{standalone}
% Page geometry
\usepackage[a4paper,margin=20mm,top=25mm,bottom=25mm]{geometry}

% Font choice
\usepackage{lmodern}

% \usepackage{lipsum}

% Use IEEE bibliography style
\bibliographystyle{IEEEtran}

% Line spacing
\usepackage{setspace}
\setstretch{1.20}

% Ensure UTF8 encoding
\usepackage[utf8]{inputenc}

% Language standard (not too important)
\usepackage[english]{babel}

% Skip a line in between paragraphs
\usepackage{parskip}

% For the creation of dummy text
\usepackage{blindtext}

% Math
\usepackage{amsmath}

% Header & Footer stuff
\usepackage{fancyhdr}
\pagestyle{fancy}
\fancyhead{}
\fancyhead[R]{\nouppercase{\rightmark}}
\fancyfoot{}
\fancyfoot[C]{\thepage}
\renewcommand{\headrulewidth}{0.0pt}
\renewcommand{\footrulewidth}{0.0pt}
\setlength{\headheight}{13.6pt}

% Epigraphs
\usepackage{epigraph}
\setlength\epigraphrule{0pt}
\setlength{\epigraphwidth}{0.65\textwidth}

% Colour
\usepackage{color}
\usepackage[usenames,dvipsnames]{xcolor}

% Hyperlinks & References
\usepackage{hyperref}
\definecolor{linkColour}{RGB}{77,71,179}
\hypersetup{
    colorlinks=true,
    linkcolor=linkColour,
    filecolor=linkColour,
    urlcolor=linkColour,
    citecolor=linkColour,
}
\urlstyle{same}

% Automatically correct front-side quotes
\usepackage[autostyle=false, style=ukenglish]{csquotes}
\MakeOuterQuote{"}

% Graphics
\usepackage{graphicx}
\graphicspath{{Images/}{../Images/}}
\usepackage{makecell}
\usepackage{transparent}

% SI units
\usepackage{siunitx}

% Microtype goodness
\usepackage{microtype}

% Listings
\usepackage[T1]{fontenc}
\usepackage{listings}
\usepackage[scaled=0.8]{DejaVuSansMono}

% Custom colours for listings
\definecolor{backgroundColour}{RGB}{250,250,250}
\definecolor{commentColour}{RGB}{73, 175, 102}
\definecolor{identifierColour}{RGB}{196, 19, 66}
\definecolor{stringColour}{RGB}{252, 156, 30}
\definecolor{keywordColour}{RGB}{50, 38, 224}
\definecolor{lineNumbersColour}{RGB}{127,127,127}
\lstset{
  language=Matlab,
  captionpos=b,
  aboveskip=15pt,belowskip=10pt,
  backgroundcolor=\color{backgroundColour},
  basicstyle=\ttfamily,%\footnotesize,        % the size of the fonts that are used for the code
  breakatwhitespace=false,         % sets if automatic breaks should only happen at whitespace
  breaklines=true,                 % sets automatic line breaking
  postbreak=\mbox{\textcolor{red}{$\hookrightarrow$}\space},
  commentstyle=\color{commentColour},    % comment style
  identifierstyle=\color{identifierColour},
  stringstyle=\color{stringColour},
   keywordstyle=\color{keywordColour},       % keyword style
  %escapeinside={\%*}{*)},          % if you want to add LaTeX within your code
  extendedchars=true,              % lets you use non-ASCII characters; for 8-bits encodings only, does not work with UTF-8
  frame=single,	                   % adds a frame around the code
  keepspaces=true,                 % keeps spaces in text, useful for keeping indentation of code (possibly needs columns=flexible)
  morekeywords={*,...},            % if you want to add more keywords to the set
  numbers=left,                    % where to put the line-numbers; possible values are (none, left, right)
  numbersep=5pt,                   % how far the line-numbers are from the code
  numberstyle=\tiny\color{lineNumbersColour}, % the style that is used for the line-numbers
  rulecolor=\color{black},         % if not set, the frame-color may be changed on line-breaks within not-black text (e.g. comments (green here))
  showspaces=false,                % show spaces everywhere adding particular underscores; it overrides 'showstringspaces'
  showstringspaces=false,          % underline spaces within strings only
  showtabs=false,                  % show tabs within strings adding particular underscores
  stepnumber=1,                    % the step between two line-numbers. If it's 1, each line will be numbered
  tabsize=2,	                   % sets default tabsize to 2 spaces
  %title=\lstname                   % show the filename of files included with \lstinputlisting; also try caption instead of title
}

% Caption stuff
\usepackage[hypcap=true, justification=centering]{caption}
\usepackage{subcaption}

% Glossary package
% \usepackage[acronym]{glossaries}
\usepackage{glossaries-extra}
\setabbreviationstyle[acronym]{long-short}

% For Proofs & Theorems
\usepackage{amsthm}

% Maths symbols
\usepackage{amssymb}
\usepackage{mathrsfs}
\usepackage{mathtools}

% For algorithms
\usepackage[]{algorithm2e}

% Spacing stuff
\setlength{\abovecaptionskip}{5pt plus 3pt minus 2pt}
\setlength{\belowcaptionskip}{5pt plus 3pt minus 2pt}
\setlength{\textfloatsep}{10pt plus 3pt minus 2pt}
\setlength{\intextsep}{15pt plus 3pt minus 2pt}

% For aligning footnotes at bottom of page, instead of hugging text
\usepackage[bottom]{footmisc}

% Add LoF, Bib, etc. to ToC
\usepackage[nottoc]{tocbibind}

% SI
\usepackage{siunitx}

% For removing some whitespace in Chapter headings etc
\usepackage{etoolbox}
\makeatletter
\patchcmd{\@makechapterhead}{\vspace*{50\p@}}{\vspace*{-10pt}}{}{}%
\patchcmd{\@makeschapterhead}{\vspace*{50\p@}}{\vspace*{-10pt}}{}{}%
\makeatother

% Tables
\usepackage{multirow}
\usepackage{longtable}
\usepackage{tabularx}
\makenoidxglossaries

\newacronym{radar}{RADAR}{Radio Detection and Ranging}
\begin{document}
\ifstandalone
\tableofcontents
\fi
% ----------------------------------------------------
\chapter{Literature Review \label{ch:literature}}
\epigraph{If you wish to make an apple pie from scratch, you must first invent the universe.}%
    {\emph{---Carl Sagan}}
\vspace{0.5cm}
% ----------------------------------------------------

\section{Integration of camera traps with the Internet of Things (IOT)}

In a paper on the use of the Raspberry Pi for biologists, Jolles \cite{Jolles-Broad-scale} argues that "[T]he Raspberry Pi [is] a great reserach tool that can be used for almost anything. This can range from ... video recording of laboratory experiments, to long term field measurement stations". The wide versatility of a device like the Raspberry Pi makes it ideal for implementation of customisable video monitoring solutions. Jolles \cite{Jolles-Broad-scale} also argues that the small footprint of the Rasbperry Pi makes it aids in its used in research, and versatile options for the provision of power allow it to be used with minimal human intervention over long time periods. These benefits align strongly with the needs of the project at hand, further emphasizing the Raspberry Pi as a viable platform around which to base the project.

Data storage and retrieval has been identified as a key limitation in the application of camera traps to monitor Red-Winged starlings. Prinz et al. \cite{Prinz-ANovel} implement a project for video surveillance of Acorn Woodpecker nests. They use the Raspberry Pi as useful resource owing to its small size and variety of storage and connectivity options. The particular project uses a WiFi connection to upload footage to cloud storage on Google Drive. They further identify that the onboard SD card has sufficient storage for around 3 days worth of footage, but also identify the strength of the Raspberry Pi in its ability to leverage 3G/4G connectivity or external flash drives and hard drives as alternative means of data storage.  The variety of footage collection and/or storage options make it likely that a suitable option can be found in the case of Red-Winged starlings.

Jolles \cite{Jolles-Broad-scale} acknowledges the existance of other microcomputer systems. However, he argues that the use of these systems is hampered by lack of hardware and software maintanence, alongside generally poorer user support compared to the widespread Raspberry Pi. 

Camacho et al. \cite{Camacho-DeploymentOf} developed a custom embedded system based on the Arduino Mega 2650 and the ATMEGA2650 chip. A custom board was favoured to allow key device peripherals to be shut down, which otherwise drew unnecessary power on the standard Arduino board. The ATMEGA2650 was identified as standing out for "the number of available pins; I2C, UART and SPI communication buses; and support from Arduino" \cite{Camacho-DeploymentOf}. This work indicates that commonly available prototyping boards should be used with caution owing to limitations such as excess power draw. Additionally, the ATMEGA2650 is identified as a viable chip for use in embedded camera trap applications, owing to its support of commonly available communication buses and support within the Arduino framework. Cardose et al. \cite{Cardoso-InternetOf} support the viability of microcontrollerbased solutions. They \cite{Cardoso-InternetOf} argue that "[A]lthough [the device is] low cost, [it] allow[s] the obtaining of images with enough quality to not influence the performance of the CV models." Microcontroller based solutions may offer more customisability and lower costs compared to microcomputer solutions such as the Raspberry Pi.

Jolles \cite{Jolles-Broad-scale} identifies that microcontrollers may have even less stringent power requirements than the Raspberry Pi, agreeing with the findings from Camacho et al. \cite{Camacho-DeploymentOf}. Jolles \cite{Jolles-Broad-scale} argues that simple, repetitive tasks are often better suited to microcontrollers. The paper also identifies that microcontrollers and microcomputers can be used in tandem in some instaces. This would allow the benefits of both systems to be realised, such as allowing for exceptionally low power draw the majority of the time, but allowing for the advanced connectivity options of the Raspberry Pi when required.


% \section{Motion detection for camera traps}


% ----------------------------------------------------
\ifstandalone
\bibliography{../Bibliography/References.bib}
\printnoidxglossary[type=\acronymtype,nonumberlist]
\fi
\end{document}
% ----------------------------------------------------